Die 81. Zusammenkunft aller deutschsprachigen Physikfachschaften (kurz: ZaPF) tagte vom 31. Oktober bis zum 3. November in Freiburg im Breisgau. Die ZaPF ist die Bundesfachschaftentagung der Physik und versteht sich dabei als eine grundlegende Basis zum Austausch zwischen den Physikfachschaften im deutschsprachigen Raum über hochschulpolitische Themen. Darüber hinaus dient sie als Gremium der Meinungsbildung und -äußerung der Physikstudierenden.

Sie tagt einmal pro Semester an unterschiedlichen Hochschulen, wobei sie von der Physikfachschaft der ausrichtenden Hochschule selbst organisiert wird. 
Im Wintersemester 2019/20 wurde die ZaPF von der Fachschaft Physik der Universität Freiburg geplant und durchgeführt.  
Es nahmen ca. 190 Fachschaftler*innen aus insgesamt 47 Fachschaften teil.\footnote{Offensichtlich noch herausfinden}
Diese tauschten sich in über 42 \footnote{Ich habe den AK Fiderallala nicht mitgezählt.} Arbeitskreisen aus und erarbeiteten Positionen zu verschiedenen Themen.

Bei der diessemestrigen Tagung waren folgende Themen Schwerpunkt: Wissenschaftskommunikation, die Nationale Forschungsdatenbank Infrastruktur (NFDI), Nachhaltigkeit, Anforderungen an Bibliotheken, Nachhaltigkeit, Solidarisierung mit den Friday for Futures, das Studienreformforum, Akkreditierungsrichtlinien und die Studienfinanzierung. \footnote{Dieser Abschnitt sollte nach Fertigstellung der Sections noch mal überarbeitet werden.}

Während der ZaPF wurden die Akkreditierungsrichtlinien weiter überarbeitet und verabschiedet. Die ZaPF beschloss zudem, als erstzeichnende Organisation des offenen Briefes zur Novellierung des Hochschulgesetzes in NRW, welcher von dem Bündnistreffen bestehend aus NRW-Fachschaften, der GEW und dem SDS verfasst wurde, aufzutreten. Für die kommende DPG Didaktik Frühjahrstagung wird beschlossen, gemeinsam mit der KFP und der JDPG als Mitorganisator des Studienreform-Forums aufzutreten. 

\section*{Bibliotheks- und Raumentwicklung}
Die Digitalisierung hat einen großen Einfluss auf die aktuelle Raum- und Bibliotheksplanung. Es gibt große Umbrüche, die Strukturen, die über Jahre aufgebaut wurden, schnell verändern. So wird kritisiert, dass Printmedien in Bibliotheken immer weiter reduziert und teilweise dezentrale Bibliotheken geschlossen werden. In dem Positionspapier \glqq Bibliotheks- und Raumentwicklung\grqq{} wird argumentiert, warum diese Strukturen erhalten bleiben und ausgebaut werden sollen.

Weiter spricht sich die ZaPF für mehr Lernräume in unterschiedlichen Formen aus. Dies fördert selbstständiges Lernen und Austausch unter Studierenden. Diese Lernräume sollten ausreichend mit Strom und Internet ausgestattet sein und möglichst rund um die Uhr zugänglich sein. Diese Forderungen sind in der Resolution \glqq Lern- und Arbeitsräume \grqq{} zu finden.

\section*{NFDI}
Die ZaPF hat sich mit der Einrichtung einer Nationalen Forschungsdateninfrastruktur (NFDI) auseinander gesetzt und befürwortet die Einrichtung dieser. In dem Positionspapier zur NFDI wird die studentische Perspektive zur Debatte, welche durch die DPG geleitet wird, festgehalten.\\
So wird gefordert, den Umgang mit der NFDI schon frühzeitig im Studium zu lehren und einen freien Zugang zu den Daten zu ermöglichen.

\section*{Fächerkombinationen im Lehramtsstudium}
Die ZaPF fordert die Studierbarkeit von mehr Fächerkombinationen als die von Mathematik und Physik indem man mehr mathematische Grundlagen in den Physikteil des Lehramtstudiums integriert. Oftmals wird Wissen aus Mathematik Vorlesungen vorausgesetzt, das Lehramtsstudierende anderer Fächerkombinationen nicht besitzen. Gegen diese Einschränkung in der Studienwahl spricht sich die ZaPF in dem Positionspapier zum Lehramt. 

\section*{Anpassung der Semesterzeiten}
In dem Resolution zur Anpassung der Semesterzeiten schließt sich die ZaPF der \glqq Empfehlung zur Harmonisierung der Semester- und Vorlesungszeiten an Deutschen Hochschulen im Europäischen Hochschulraum \grqq{} an. Damit wird eine weitere Internationalisierung der deutschen Hochschulen gefordert.\\
Mit dieser Resolution wird die Resolution zu Internationalen Semesterzeiten bestärkt, die auf der ZaPF im Sommersemester 2016 in Rostock verabschiedet wurde. Sie wurde gemeinsam mit den BuFaTas der ... ausgearbeitet und beschlossen.\footnote{Hier muss Vicky helfen}

\section*{Prüfungsunfähigkeitsbescheinigungen}
Auch diese Resolution baut auf einer bereits verabschiedeten Resolution zu Symptompflicht auf Attesten, welche auf der ZaPF im Wintersemester 2016/17 in Dresden beschlossen wurde. Mit den BuFaTas ... eine gemeinsame Resolution ausgearbeitet und verabschiedet. \footnote{Hier muss Vicky helfen.} \\
Es wird sich dafür ausgesprochen, eine Prüfungsunfähigkeitsbescheinigung als solche anzuerkennen und nicht zu verlangen, persönliche Symptome preisgeben zu müssen.

\section*{Solidarisierung mit den Fridays for Futures} 
Die Bewegung Fridays for Future setzt sich für die Anerkennung wissenschaftlicher Tatsachen bezüglich des Kilmawandels innerhalb der Bevölkerung ein. Deswegen spricht sich die ZaPF in der Resolution Solidarisierung mit den Fridays for Future dafür aus, dass an Hochschulen die Freiräume geschaffen werden, dass (junge) Erwachsene an den Protesten teilnehmen können. Weiter spricht sie sich gegen jegliche Repressionen gegen die an den Protesten teilnehmenden Menschen aus. 

\section*{Wissenschaftskommunikation}
Die ZaPF hat sich mit der Rolle der Wissenschaftskommunikation auseinander gesetzt und Forderungen in der Resolution zu Wissenschaftskommunikation festgehalten. So fordert die ZaPF, Wissenschaftskommunikation in den Curriculum zu integrieren und aktiv zu fördern. Hierfür empfiehlt die ZaPF. \footnote{Vielleicht kann hier jemand einen kreativen Satz über die Rolle im Studium schreiben.}

AKs, die es bisher noch nicht in den Bericht geschafft haben:
\begin{itemize}
	\item CHE Ranking
	\item Kritische Physik
	\item Open Science
	\item Open Access
	\item BAföG
	\item Studienfinanzierung
	\item Nachhaltigkeit
	\item Studienreform Forum
	\item Ethik
	\item BaMa-Umfrage
	\item Awareness-Konzepte
	\item Gleichstellung
	\item Modernisierung
	\item Datenschutz
\end{itemize}
