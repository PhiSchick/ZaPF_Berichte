Die 81. Zusammenkunft aller deutschsprachigen Physikfachschaften (kurz: ZaPF) tagte vom 31. Oktober bis zum 3. November in Freiburg im Breisgau. Die ZaPF ist die Bundesfachschaftentagung der Physik und versteht sich dabei als eine grundlegende Basis zum Austausch zwischen den Physikfachschaften im deutschsprachigen Raum über hochschulpolitische Themen. Darüber hinaus dient sie als Gremium der Meinungsbildung und -äußerung der Physikstudierenden.

Die ZaPF tagt einmal pro Semester an unterschiedlichen Hochschulen, wobei sie von der Physikfachschaft der ausrichtenden Hochschule selbst organisiert wird. 
Im Wintersemester 2019/20 wurde die ZaPF von der Fachschaft Physik der Universität Freiburg geplant und durchgeführt.  
Es nahmen ca. 214 Fachschaftler*innen aus insgesamt 56 Fachschaften teil.
Diese tauschten sich in über 42 Arbeitskreisen aus und erarbeiteten Positionen zu verschiedenen Themen.

Bei der Tagung im Wintersemester 2019/20 waren folgende Themen Schwerpunkt: Wissenschaftskommunikation, die nationale Forschungsdateninfrastruktur (NFDI), Nachhaltigkeit, Anforderungen an Bibliotheken, Solidarisierung mit Fridays for Future, Datenschutz und die Studienfinanzierung. 

Während der ZaPF wurden die Akkreditierungsrichtlinien weiter überarbeitet und verabschiedet. Die ZaPF beschloss zudem, als erstzeichnende Organisation des offenen Briefes zur Novellierung des Hochschulgesetzes in NRW, welcher von dem Bündnistreffen bestehend aus NRW-Fachschaften, der GEW und dem SDS verfasst wurde, aufzutreten. Darüber hinaus wurde beschlossen, als Veranstalter des Studienreform-Forums aufzutreten, welches im Frühjahr 2019 im Rahmen der DPG-Frühjahrstagung in Aachen stattgefunden hatte. Im Rahmen der DPG-Frühjahrstagung in Bonn soll es wieder stattfinden.

\section*{Wissenschaftskommunikation}
Die ZaPF hat sich mit der Rolle der Wissenschaftskommunikation auseinander gesetzt und Forderungen in der Resolution zu Wissenschaftskommunikation\footnote{\url{www.zapfev.de/resolutionen/wise19/wissenschaftskommunikation/wissenschaftskommunikation.pdf}} festgehalten. Im Einzelnen fordert die ZaPF eine Implementation in die Lehre und empfiehlt hierfür zum Beispiel eine fakultätsübergreifende Veranstaltung oder die Möglichkeit, seine Abschlussarbeit vor fachfremden Publikum vorzustellen, sowie die Förderung von Wissenschaftskommunikation durch die Hochschulen.

\section*{Bibliotheks- und Raumentwicklung}
Die Raum- und Bibliotheksplanung an den Universitäten erlebt - nicht zuletzt aufgrund der Digitalisierung - derzeit starke Umbrüche. Die ZaPF kritisiert den Trend, dass die Verfügbarkeit von Printmedien in Bibliotheken immer weiter reduziert und teilweise dezentrale Bibliotheken geschlossen werden. In dem Positionspapier zur Bibliotheks- und Raumentwicklung wird argumentiert, warum diese Strukturen erhalten bleiben und ausgebaut werden sollen.

Weiter spricht sich die ZaPF für mehr Lernräume in unterschiedlichen Formen aus. Dies fördert selbstständiges Lernen und Austausch unter Studierenden. Diese Lernräume sollten ausreichend mit Strom und Internet ausgestattet sein und möglichst rund um die Uhr zugänglich sein. Diese Forderungen sind in der Resolution zu Lern- und Arbeitsräume zu finden.

\section*{NFDI}
Die ZaPF hat sich mit der Einrichtung einer Nationalen Forschungsdateninfrastruktur (NFDI) auseinander gesetzt und befürwortet die Einrichtung dieser. In dem Positionspapier zur NFDI\footnote{\url{www.zapfev.de/resolutionen/wise19/nfdi/nfdi.pdf}} werden Anforderungen an eine solche Infrastruktur aus der Sicht von Physikstudierenden festgehalten.

Das Positionspapier beinhaltet fünf Abschnitte, welche im Einzelnen die Integration der verschiedenen Disziplinen innerhalb der Physik, die Einbindung in die Lehre, den freien Zugang, die angebotenen Dienste und deren Sicherheit sowie die Schnittstellen und Struktur thematisieren.

\section*{Fächerkombinationen im Lehramtsstudium}
Die ZaPF fordert die Studierbarkeit von mehr Fächerkombinationen als die von Mathematik und Physik indem man mehr mathematische Grundlagen in den Physikteil des Lehramtstudiums integriert. Häufig wird Wissen aus Mathematikveranstaltungen vorausgesetzt, das Lehramtsstudierende anderer Fächerkombinationen nicht besitzen. Gegen diese Einschränkung in der Studienwahl spricht sich die ZaPF in dem Positionspapier zum Lehramt aus. 

\section*{Anpassung der Semesterzeiten}
In der Resolution zur Anpassung der deutschen Hochschulen an internationale Semesterzeiten schließt sich die ZaPF der \glqq Empfehlung zur Harmonisierung der Semester- und Vorlesungszeiten an Deutschen Hochschulen im Europäischen Hochschulraum \grqq{} der Hochschulrektorenkonferenz an. Mit dieser Anpassung soll die Studierendenmobilität gefördert werden.
Mit dieser Resolution wird die Resolution zu Internationalen Semesterzeiten bestärkt, die auf der ZaPF im Sommersemester 2016 in Konstanz verabschiedet wurde.\\
Sie wurde als gemeinsame Resolution verschiedener BuFaTas ausgearbeitet und wurde bisher durch die  Bundesfachschaftstagungen der Wirtschafts- und  Wirtschaftssozialwissenschaften  (Bufak Wiso), der Mathematik (Koma), der Psychologie  (Psyfako), Elektrotechnik (BuFaTa ET), Geographie (GeoDACH) unterstützt.

\section*{Prüfungsunfähigkeitsbescheinigungen}
Auch diese Resolution baut auf einer bereits verabschiedeten Resolution zu Symptompflicht auf Attesten, welche auf der ZaPF im Wintersemester 2016/17 in Dresden beschlossen wurde. 
Es wird sich  dafür ausgesprochen, eine Prüfungsunfähigkeitsbescheinigung als solche  anzuerkennen und nicht zu verlangen, persönliche Symptome preisgeben zu  müssen.

Sie wurde als gemeinsame Resolution verschiedener BuFaTas ausgearbeitet und wurde bisher durch die  Bundesfachschaftstagungen der Wirtschafts- und   Wirtschaftssozialwissenschaften  (Bufak Wiso), der Mathematik (Koma),  der Psychologie  (Psyfako), Elektrotechnik (BuFaTa ET), Geographie (GeoDACH) , Informatik (KIF) unterstützt.

\section*{Solidarisierung mit den Fridays for Future} 
Die ZaPF solidarisiert sich mit den aktuellen Forderungen der Bewegung \glqq{}Fridays for Future\grqq{}. Insbesondere befürwortet sie das Bestreben der \glqq Fridays for Future \grqq{},  Akzeptanz für wissenschaftlich Tatsachen in der Gesellschaft zu verbreiten.\\
Weiterhin spricht sie sich dafür aus, dass an Schulen und Hochschulen die Freiräume geschaffen werden, dass Kinder, Jugendliche und Erwachsene an den Protesten teilnehmen können. Außerdem verurteilt sie jegliche Repressionen gegen die an den Protesten teilnehmenden Menschen aus. 

